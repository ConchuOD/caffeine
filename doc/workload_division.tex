\documentclass[a4paper, 11pt]{article}
\usepackage[margin=2.5cm]{geometry}

\usepackage{enumitem}
\usepackage{calc}
\usepackage{amsmath}
\usepackage{amssymb}
\usepackage{courier}

\title{Workload division}
\begin{document}

\maketitle

A cursory glance at the number of commits belonging to both of us might suggest a large imbalance of the amount of work performed by each team member - Conor's commits contribute more than three times the amount attributed to Andrew. While overall, Conor may have contributed more commits, we feel that this is explained by our respective approaches to version control committing. Conor has the approach of committing each small feature addition/refactor separately, while Andrew combines several deltas into a single commit, in the main. Conor's approach is probably more appropriate, however.

The work distribution was fair and equitable between us, with development and testing effort significant from both of us. Refactoring efforts from us both can be clearly seen in the number of lines that have been removed from the project by both of us, as we put a large emphasis on being able to reuse code where possible.

We communicated almost daily regarding this project both in person, using text, and in video chat when working from home. This consistent effort to raise problems, difficulties, and progress as code was architected and developed lead to a large proportion of the project specification being completed in comfortable time. We enjoyed working in a team together for this project. 

Early on, we adopted the approach of having to justify to the other team member the reasons for which they had chosen a particular solution to a problem. This forced us to continually evaluate our approaches, to align with good software development practises as described in the lecture notes, and from previous professional work experience.

We attempted to follow certain principles of the Agile methodology where possible. We used the feature chart on GitHub to table functionality to be implemented, and update its status as moved to being implemented, to testing. We used the Issues functionality on GitHub to raise attention to bugs in the code, finishing with almost 70 raised and resolved issues. 

We performed brief retrospectives at certain stages of the project (before the scheduled code reviews mostly), discussing which features we would implement differently, and noting what was implemented in a way which we had found to be easily extensible and understandable.
	
\end{document}